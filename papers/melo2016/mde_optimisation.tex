\section{Model-based Optimisation}
\label{section:mde_optimisation}

	%\begin{draftlist}
		%Introduce the mechanism:
		%\item Basic concepts: What are the key ingredients of search-based/optimisation-based techniques? Start solution(s), Finding new solutions from existing
					%ones, Evaluating solution fitness. On this basis we can generically implement `standard' optimisation algorithms.
		%\item Explain what these ingredients can be in an MDE context. Start solutions: Models (discuss different sources for these models); New solutions:
					%transformations and merging; Fitness: queries, simulations, \ldots. 
		%\item Present the language and tool for expressing this.
	%\end{draftlist}

	Three ingredients are required for any search-based algorithm: 
	\begin{enumerate}
		\item A representation of individual candidate solutions;
		\item A mechanism for generating new candidate solutions from existing candidate solutions (e.g., through mutation or breeding); and
		\item A mechanism for evaluating the quality of candidate solutions; that is how well they satisfy each of the optimisation objectives (often called the solution's fitness).
	\end{enumerate}
	Most search algorithms also require a means of generating an initial population of candidate solutions. Once these ingredients have been defined for a specific problem, we can apply standard 
	search-based algorithms.
	
	As discussed above, we will use models to represent individual candidate solutions. Therefore, the overall search space is defined by providing a meta-model. An initial population of candidate
	solutions can be provided in a number of ways---for example, it could be provided as a set of explicit model files or we could use constraint solvers like Alloy \cite{generating_models} to generate
	a suitable set of initial models.
	
	To generate new solutions from existing ones, endogenous model transformations are an obvious candidate. In particular, we propose to use graph transformations, as they have a clear and simple
	syntax for easily expressing endogenous transformations. For example, Fig.~\figref{zoo_rule} shows a simple Henshin rule that can be used for the search problem described in the previous section.
	Because these rules are defined on the model level, we will often be able to easily write them in a way that ensures all well-formedness rules are preserved.
	
	\insertFigure[caption={Henshin rule for moving animals between cages}]{zoo_rule}
	
	Evaluating the fitness of candidate solutions can take many different forms. In the simplest case, fitness may be determined by a model query---for example expressed in OCL. In other cases, we may
	require to run a simulation of the candidate solution, which may involve further transformations etc. (e.g., \cite{Efstathiou+14,Chatziprimou+14}).
	
	To test out these ideas, we are currently developing a prototype tool for model optimisation.\footnote{See \url{https://github.com/szschaler/mde_optimiser}} Our tool provides a simple Xtext-based
	DSL to allow describing model-based search problems together with an interpreter for running searches. Search algorithms, fitness functions, and initial model provision are modularised behind Java
	interfaces. For search algorithms this means that it is easy to incorporate existing implementations, such as the MOEA framework\footnote{See \url{http://moeaframework.org/}}. We currently have no
	DSL-level support for fitness functions and initial model generation, but plan to add these features. For now, they are specified by providing Java implementations. Solution evolution is realised by
	Henshin transformations.
	
	\draft{Consider showing example syntax for Zoo case.}