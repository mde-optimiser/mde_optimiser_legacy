\section{Related Work}
\label{section:related_work}

	\begin{draftlist}
		To discuss:
		\item MOMOT \cite{Fleck15}
		\item QVTR \cite{Drago+10,Drago+11,Drago+15}
		\item Crepe \cite{Efstathiou+14b,Williams13}
		\item Searching Models, Modeling Search: On the Synergies of SBSE and MDE \cite{Kessentini+13}. General position paper. Focuses on genetic encoding of models 
		      (like Crepe). Proposes a general infrastructure/architecture for MDE search.
		\item \cite{Denil+14} (also a version at https://www.cs.mcgill.ca/files/techReports/icse.pdf) use transformation scheduling specifications to implement 
		      single-state (non-population based) optimisation algorithms directly over models 
		      (using transformations to describe exploration steps). They provide some good arguments for why this is a good approach. I think their approach would be
					difficult to extend to population-based approaches as the current model is somewhat implicit in the scheduling specification. Also, they do not touch on
					what breeding means for models. They provide some performance analysis.
					
					%Look at references in \cite{Denil+14}!
					
		\item \cite{Abdeen+14} present an approach similar to MOMOT, but based on Viatra. Have some interesting discussion of repair/ranking of invalid solution
		      candidates.
					
	  \item \cite{BurtonPoulding13} gives a good discussion of why one would want to optimise at the model level. They also introduce the idea of domain-specific
		      encodings (even though they don't call it that). In \cite{Burton+12} they give a concrete application example (resource allocation) and present an 
					Epsilon-based tool doing the optimisation. An interesting effect here is that they are specifically identifying only a part of the model that the optimiser
					actually manipulates (correspondences between model elements in a requirements and an implementations model). This allows crossover to be expressed
					relatively easily, but comes at the cost of having to run a transformation for every fitness evaluation to reconstruct the ``phenotype'' model.
	\end{draftlist}
