\section{Related Work}
\label{section:related_work}

	%\begin{draftlist}
		%To discuss:
		%\item MOMOT \cite{Fleck15}
		%\item QVTR \cite{Drago+10,Drago+11,Drago+15}
		%\item Crepe \cite{Efstathiou+14b,Williams13}
		%\item Searching Models, Modeling Search: On the Synergies of SBSE and MDE \cite{Kessentini+13}. General position paper. Focuses on genetic encoding of models 
		      %(like Crepe). Proposes a general infrastructure/architecture for MDE search.
		%\item \cite{Denil+14} (also a version at https://www.cs.mcgill.ca/files/techReports/icse.pdf) use transformation scheduling specifications to implement 
		      %single-state (non-population based) optimisation algorithms directly over models 
		      %(using transformations to describe exploration steps). They provide some good arguments for why this is a good approach. I think their approach would be
					%difficult to extend to population-based approaches as the current model is somewhat implicit in the scheduling specification. Also, they do not touch on
					%what breeding means for models. They provide some performance analysis.
					%
					%%Look at references in \cite{Denil+14}!
					%
		%\item \cite{Abdeen+14} present an approach similar to MOMOT, but based on Viatra. Have some interesting discussion of repair/ranking of invalid solution
		      %candidates.
					%
	  %\item \cite{BurtonPoulding13} gives a good discussion of why one would want to optimise at the model level. They also introduce the idea of domain-specific
		      %encodings (even though they don't call it that). In \cite{Burton+12} they give a concrete application example (resource allocation) and present an 
					%Epsilon-based tool doing the optimisation. An interesting effect here is that they are specifically identifying only a part of the model that the optimiser
					%actually manipulates (correspondences between model elements in a requirements and an implementations model). This allows crossover to be expressed
					%relatively easily, but comes at the cost of having to run a transformation for every fitness evaluation to reconstruct the ``phenotype'' model.
	%\end{draftlist}

	Table~\ref{table:related_work} summarises the current literature on optimisation in MDE.\footnote{Some other approaches exist, but they have only been defined
	for a specific problem. Here, we focus on approaches that aim to be generic.} We use two orthogonal sets of categories for characterising the different
	approaches that have been explored so far:
	\begin{enumerate}
		\item \emph{By optimisation target.} Some approaches focus on selecting optimal \emph{transformations} (producing optimal models), while other
		      approaches focus on finding optimal \emph{models} directly without considering optimality of transformations (and, possibly, without
					explicitly specified transformations).
		\item \emph{By encoding approach.} Two general approaches have emerged: (1) general encoding of MDE artefacts using standard genetic
		      encodings and applying genetic optimisation algorithms, and (2) techniques that work directly on the structure of the MDE artefacts themselves (possibly
					with annotations).
	\end{enumerate}
	We will briefly discuss these categories in more detail below.
	
	\begin{table}[tbp]
		\centering
			\begin{tabular}{r|cc}
				                & Optimisation of models                         & Optimisation of transformations \\
				\hline 
				Genome encoding & \cite{Efstathiou+14b,Williams13,Kessentini+13} & \cite{Fleck+15,Abdeen+14} \\
				Direct search   & \cite{Denil+14,Burton+12,BurtonPoulding13}     & \cite{Drago+10,Drago+11,Drago+15} \\
			\end{tabular}
		\caption{Overview of related work}
		\label{table:related_work}
	\end{table}
	

	\subsection{Optimisation of transformations}

		A number of authors have proposed approaches that aim to search for optimal transformations instead of looking only for optimal models. Optimality of the
		models produced is still of interest, but they are only indirectly manipulated by searching the space of viable model transformations. This indirect approach
		is chosen for one of two reasons: 
		\begin{enumerate}
			\item There may be optimality criteria directly on the transformations themselves. For example, when optimising cloud data centre configurations based on a
			      current configuration, we are not only interested in finding the optimal new configuration, but also the shortest or least costly path there.
						Searching for transformations rather than for models directly allows these objectives to be expressed and incorporated into the search process.
		  \item The transformations encode developer choices, but may need to be rerun for other source models at a later stage. In other words, the transformations
			      are the actual optimisation object, their application to models is separate to the optimisation process.
		\end{enumerate}
			
		\textbf{Using genome encodings.}
	%
			%\begin{draftlist}
				%To discuss:
				%\item MOMOT \cite{Fleck+15}					
				%\item \cite{Abdeen+14} present an approach similar to MOMOT, but based on Viatra. Have some interesting discussion of repair/ranking of invalid solution
							%candidates.
			%\end{draftlist}
			%
			Abdeen \emph{et al.} \cite{Abdeen+14} propose encoding different transformation sequences as genomes so that they become amenable to search
			with a genetic algorithm. Their approach is based on Viatra. Each candidate solution is a tuple made from a start model and a sequence of applications of
			small, predefined Viatra transformations. The paper provides some good discussion of the issue of invalid individuals: these may be the result of crossover
			or mutation, making transformation chains invalid or causing them to create invalid models. \cite{Abdeen+14} uses a specific ranking mechanism to handle
			these cases.
			
			Fleck \emph{et al.} \cite{Fleck+15}, propose MOMOT. MOMOT uses base transformations expressed in Henshin and optimises chains of applications of these
			transformations to a base model. Fleck \emph{et al.} handle invalid candidate solutions with a repair mechanism, effectively reducing the population size
			for these situations.
		
		\textbf{Using direct search.}
		%
		  %\begin{draftlist}
			  %To discuss:
				%\item \cite{Drago+10,Drago+11,Drago+15}
			%\end{draftlist}
			Drago \emph{et al.} \cite{Drago+10,Drago+11,Drago+15} proposed QVT-R$^2$, an extension of QVT-Relational. In QVT-R$^2$, transformation developers create 
			non-deterministic transformations by providing multiple rules matching the same structure in the source model. The rules are then annotated with information
			about relevant quality analysis (e.g., information about how to invoke a an external performance analysis). Each choice point is then incrementally and
			interactively fixed for a given source model by running all transformation options and asking the developer to choose the best resulting model based on the
			automatically invoked quality analyses.

	\subsection{Optimisation of models}
	
	  \textbf{Using genome encodings.}
		%
			%\begin{draftlist}
				%To discuss:
				%\item Crepe \cite{Efstathiou+14b,Williams13}
				%\item Searching Models, Modeling Search: On the Synergies of SBSE and MDE \cite{Kessentini+13}. General position paper. Focuses on genetic encoding of 
				      %models (like Crepe). Proposes a general infrastructure/architecture for MDE search.
			%\end{draftlist}
			%
			Williams \cite{Williams13} proposes Crepe, a generic encoding of models as sequences of integers. Given a meta-model and a so-called ``finitisation model'' 
			(containing information about the range of attributes), Crepe provides a unique, bi-directional transformation between instances of the meta-model and
			integer-vector representations of these models. The integer vectors can be used as genomes in the context of genetic algorithms using standardised, 
			domain-independent operators for mutation and crossover. In \cite{Efstathiou+14b}, the authors provide some extensions to the approach as well as a first
			comparative evaluation of its performance compared to a manually implemented optimisation.
			
			Kessentini \emph{et al.} \cite{Kessentini+13}, give a high-level overview of an architecture for reusing SBSE techniques in the context of MDE. A key
			ingredient of their approach is also a generic meta-model for encoding models as genomes, making them amenable to standard genetic operators such as
			mutation and crossover.

	  \textbf{Using direct search.}
%
			%\begin{draftlist}
				%To discuss:
				%\item \cite{Denil+14} (also a version at https://www.cs.mcgill.ca/files/techReports/icse.pdf) use transformation scheduling specifications to implement 
							%single-state (non-population based) optimisation algorithms directly over models 
							%(using transformations to describe exploration steps). They provide some good arguments for why this is a good approach. I think their approach would 
							%be difficult to extend to population-based approaches as the current model is somewhat implicit in the scheduling specification. Also, they do not
							%touch on what breeding means for models. They provide some performance analysis.
							%
				%\item \cite{BurtonPoulding13} gives a good discussion of why one would want to optimise at the model level. They also introduce the idea of domain-specific
							%encodings (even though they don't call it that). In \cite{Burton+12} they give a concrete application example (resource allocation) and present an 
							%Epsilon-based tool doing the optimisation. An interesting effect here is that they are specifically identifying only a part of the model that the 
							%optimiser actually manipulates (correspondences between model elements in a requirements and an implementations model). This allows crossover to be
							%expressed relatively easily, but comes at the cost of having to run a transformation for every fitness evaluation to reconstruct the ``phenotype''
							%model.
			%\end{draftlist}
			%
			Burton and Poulding \cite{BurtonPoulding13} were the first to describe an idea for running transformations directly on models. They create separate 
			domain-specific modelling languages describing a search problem and candidate solutions and run search to find near-optimal solutions. As described in 
			\cite{Burton+12}, their solutions are mappings between solution and problem models, effectively limiting the problem to the optimisation of vectors of
			binary associations. This enables them to easily define general mutation and crossover operators so that standard evolutionary search can be applied.
			
			Denil \emph{et al.} \cite{Denil+14} present a general approach for model-based optimisation using their Formalism Transformation Graph and Process Model 
			(FTG+PM). Their main focus is implementing different search algorithms as transformation scheduling programs, fully integrating them into a general MDE
			approach. They show how to implement a number of single-state search algorithms in this way, applying them to a problem in circuit design. Candidate
			solutions are implicit in the transformation scheduling language, which may make it difficult to extend this approach to population-based search algorithms.