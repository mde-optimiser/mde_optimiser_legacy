\newif\ifdraft
\drafttrue
%\draftfalse

\ifdraft
	\documentclass[draft]{llncs}
	
	\usepackage{drafts}
\else
	\documentclass{llncs}

	\usepackage[nodrafts]{drafts}
\fi

\usepackage{figures}

\usepackage{hyperref}

\title{Towards Model-Based Optimisation: Using domain knowledge explicitly}

%\titlerunning{\draft{}}  % abbreviated title (for running head) also used for the TOC unless \toctitle is used

\author{Steffen Zschaler\inst{1}}
\institute{
	Department of Informatics\\
	King's College London\\
	\email{szschaler@acm.org}
}

\begin{document}

	\maketitle
	
	\begin{abstract}
		Search-based software engineering (SBSE) aims to treat soft\-ware-design problems as search and optimisation problems and address them by applying automated
		search and optimisation algorithms. A key concern is the adequate capture and representation of the structure of these design problems. Model-driven
		engineering (MDE) has a strong focus on domain-specific languages (DSLs) which are defined through meta-models, capturing the structure and constraints of a
		particular domain. There is, thus, a clear argument for combining both techniques to obtain the best of both worlds. Some authors have proposed a number of
		approaches in recent years, but these have mainly focused on the optimisation of transformations or on the identification of good generic encodings of models
		for search.
		%
		In this paper, we first provide a structured overview of the current state of the art before identifying limitations of the key proposals (transformation
		optimisation and generic genetic encodings of models). We then present a first prototype for running search algorithms directly on models themselves (rather
		than a separate representation) and derive key research challenges for this approach to model optimisation.
		
		\ifdraft
			\draft{To be submitted to MELO. max 12 pages, LNCS. Paper: April 18, 2016. Have permission to extend DL to April 22}
		\fi
		
		\keywords{evolutionary optimisation, object space, model-driven engineering, model transformations}
	\end{abstract}
	
	\section{Introduction}
\label{section:introduction}

	%\begin{draftlist}
		%\item Search-based and optimisation approaches to design are becoming more important (SBSE), but they are also important to adaptation (cite our cloud work,
					%mention robotics). 
		%\item The domains concerned can be quite complex, capturing them in traditional optimisation-oriented ways is difficult.
		%\item Some earlier work exists that tries to define a generic mechanism of translating models into genome representation (cite Crepe complete paper and
					%James' thesis), but this has issues. Discuss these issues in detail here or possibly in its own section below.
		%\item Main problem: loses lots of domain-specific knowledge.
		%\item We have spent lots of time to design a good meta-model based on extensive domain analysis. Why not use this knowledge when developing optimisations?
		%\item In this paper, we propose a technique and prototype tool for doing so and present its application to a simple case study. This forms the basis of a 
					%research agenda.
		%\item Structure of the paper etc.
	%\end{draftlist}

	Search-based software engineering (SBSE) is about using optimisation techniques for automating the search for (near-)optimal software designs \cite{SBSE}. More
	recently, the use of search-based approaches has also been extended to software adaptation (e.g., \cite{Efstathiou+14,Chatziprimou+14}). Using search-based
	techniques allows the exploration of much larger design spaces than could be explored manually by developers. As a result, better solutions can be identified more
	quickly.
	
	However, as has been recognised before \cite{BurtonPoulding13,Kessentini+13}, the problem domains in software engineering are too complex to be effectively
	captured with traditional problem representations as they are typically used in search-based systems. Model-driven engineering (MDE) offers good techniques for
	capturing complex domains including their structural constraints by using meta-models. As a result, there has recently been increased interest in combining the
	advantages of SBSE and MDE \cite{Fleck15,Drago+10,Drago+11,Drago+15,Efstathiou+14b,Williams13,Denil+14,Abdeen+14}.
	
	Much of this work has focused on finding good generic representations of models that are tailored towards the needs of standard optimisation algorithms (most
	typically, genetic algorithms \cite{GA}). As we will discuss in detail in Sect.~\ref{section:issues}, these generic encodings introduce their own challenges.
	Most importantly, they make it easy for search steps to produce invalid candidate solution; that is, models that do not satisfy the constraints expressed by the
	meta-model or its well-formedness rules.
	
	In this paper, we propose an alternative approach: instead of defining a secondary encoding for candidate solutions, we propose to run optimisation algorithms
	directly on models represented in standard meta-modelling data structures. We argue that given that developers have spent substantial time and effort designing
	meta-models that are a good representation of the domain, we should make use of as much of this information as possible during search and optimisation. We
	present a first prototype of a tool for running such optimisations and discuss some of the research challenges that need to be addressed to make this vision a
	reality.
	
	The remainder of this paper is strucured as follows: \draft{}
	
	\section{Related Work}
\label{section:related_work}

	\begin{draftlist}
		To discuss:
		\item Look at Manuel Wimmer's paper at MOMO'16 for related work. They have a solution using MOEA and Henshin
		\item QVTR
		\item Crepe
		\item Searching Models, Modeling Search: On the Synergies of SBSE and MDE \cite{Kessentini+13}. General position paper. Focuses on genetic encoding of models 
		      (like Crepe). Proposes a general infrastructure/architecture for MDE search.
		\item \cite{Denil+14} (also a version at https://www.cs.mcgill.ca/files/techReports/icse.pdf) use transformation scheduling specifications to implement 
		      single-state (non-population based) optimisation algorithms directly over models 
		      (using transformations to describe exploration steps). They provide some good arguments for why this is a good approach. I think their approach would be
					difficult to extend to population-based approaches as the current model is somewhat implicit in the scheduling specification. Also, they do not touch on
					what breeding means for models. They provide some performance analysis.
					
					Look at references in \cite{Denil+14}!
					
		\item \cite{Abdeen+14} present an approach similar to MOMOT, but based on Viatra. Have some interesting discussion of repair/ranking of invalid solution
		      candidates.
	\end{draftlist}


	\section{Issues with Generic Encoding of Models}
\label{section:issues}

	\begin{draftlist}
		\item Describe key ideas behind Crepe's approach to encoding. 
		\item Introduce the Zoo meta-model and show an example encoding.
		\item Then show how this is broken by simple crossover.
	\end{draftlist}

	
	\section{Model-based Optimisation}
\label{section:mde_optimisation}

	%\begin{draftlist}
		%Introduce the mechanism:
		%\item Basic concepts: What are the key ingredients of search-based/optimisation-based techniques? Start solution(s), Finding new solutions from existing
					%ones, Evaluating solution fitness. On this basis we can generically implement `standard' optimisation algorithms.
		%\item Explain what these ingredients can be in an MDE context. Start solutions: Models (discuss different sources for these models); New solutions:
					%transformations and merging; Fitness: queries, simulations, \ldots. 
		%\item Present the language and tool for expressing this.
	%\end{draftlist}

	Three ingredients are required for any search-based algorithm: 
	\begin{enumerate}
		\item A representation of individual candidate solutions;
		\item A mechanism for generating new candidate solutions from existing candidate solutions (e.g., through mutation or breeding); and
		\item A mechanism for evaluating the quality of candidate solutions; that is how well they satisfy each of the optimisation objectives (often called the solution's fitness).
	\end{enumerate}
	Most search algorithms also require a means of generating an initial population of candidate solutions. Once these ingredients have been defined for a specific problem, we can apply standard 
	search-based algorithms.
	
	As discussed above, we will use models to represent individual candidate solutions. Therefore, the overall search space is defined through a meta-model. An
	initial population of candidate solutions can be provided in a number of ways---for example, it could be provided as a set of explicit model files or we could
	use constraint solvers like Alloy \cite{Jackson02} to generate a suitable set of initial models (e.g., using the Cartier tool originally developed for
	transformation testing \cite{Sen+08,Sen+09}).
	
	To generate new solutions from existing ones, endogenous model transformations are an obvious candidate. In particular, we propose to use graph transformations,
	as they have a clear and simple syntax for easily expressing endogenous transformations. For example, Fig.~\figref{zoo_rule} shows a simple Henshin rule that can
	be used for the search problem described in the previous section. Because these rules are defined on the model level, we will often be able to easily write them
	in a way that ensures well-formedness rules are preserved.
	
	\insertFigure[caption={Henshin rule for moving animals between cages}]{zoo_rule}
	
	Evaluating the fitness of candidate solutions can take many different forms. In the simplest case, fitness may be determined by a model query---for example expressed in OCL. In other cases, we may
	require to run a simulation of the candidate solution, which may involve further transformations etc. (e.g., \cite{Efstathiou+14,Chatziprimou+14}).
	
	To test out these ideas, we are currently developing a prototype tool for model optimisation.\footnote{See \url{https://github.com/szschaler/mde_optimiser}} Our tool provides a simple Xtext-based
	DSL to allow describing model-based search problems together with an interpreter for running searches. Search algorithms, fitness functions, and initial model provision are modularised behind Java
	interfaces. For search algorithms this means that it is easy to incorporate existing implementations, such as the MOEA framework\footnote{See \url{http://moeaframework.org/}}. We currently have no
	DSL-level support for fitness functions and initial model generation, but plan to add these features. For now, they are specified by providing Java implementations. Solution evolution is realised by
	Henshin transformations.
	
	Figure~\figref{zoo_code} shows an example specification of the \texttt{Zoo} example in our tool. After some configuration information in the first line, this
	code declares the structure of the search space by indicating a meta-model, and then defines relevant fitness functions and model evolvers. Fitness functions are
	currently provided by implementing a specific Java interface; we are planning to provide full OCL integration in the language for simple model queries. Evolvers
	are defined by specifying a Henshin model and naming a unit (often a rule) in this model.
	
	\insertFigure[caption={Specification of the \texttt{Zoo} search problem}, targetwidth=.8\textwidth]{zoo_code}
	
	Figure~\figref{run_code} shows how to run a search using our tool. At this point, we only support programmatic invocation. For this, a new \texttt{Interpreter}
	object needs to be created and configured with a parsed version of the problem, a \texttt{ModelProvider} for generating initial models, and a generic search
	algorithm (a variant of random hill climbing in this example). Invoking \texttt{execute} runs the search as specified and returns the set of solutions found.
	
	\insertFigure[caption={Basic code for running model-based search algorithms}, targetwidth=.9\textwidth]{run_code}
	
	
	%\section{Case Study: The Zoo Model}
\label{section:case_study}

	\begin{draftlist}
		\item Introduce possible optimisation objectives.
		\item Show a single-objective case and how it runs.
		\item Show a multi-objective case and how it runs.
	\end{draftlist}

	
	\section{Research Challenges}
\label{section:research_challenges}

	%\begin{draftlist}
		%Challenges:
		%\item Notions of neighbourhoods or directions to allow more existing algorithms to be transferred. Alternatively, find optimisation algorithms that work more
					%efficiently with the direct object-structure encoding.
		%\item Running/selecting evolvers efficiently, considering pre-conditions. Random matching in graph transformation engines.
		%\item More compact representations of evolvers, considering that large parts of the model stay constant?
		%\item Performance on large systems? Suitable for on-line adaptation?
		%\item Languages for expressing different fitness evaluations?
		%\item Flexible definition of model providers (mix concerns between needs of algorithm and needs of domain/problem).
		%\item \ldots
	%\end{draftlist}

  Our initial work has identified a number of challenges requiring further research to enable model-based optimisation to be used effectively.
	
	\subsection{Reuse of existing optimisation algorithms}
	
		Some existing optimisation algorithms make particular assumptions about the search space. For example, hill climbing, a basic single-objective search algorithm,
		expects to be able to identify the complete ``neighbourhood'' of a given candidate solution so that this can be systematically explored. Similarly, swarm-based
		search algorithms expect to be able to identify a ``direction'' vector between solutions and to use this to guide the derivation of one solution from another.
		These notions are easily defined in classic search-based approaches, where solutions are represented by (high dimensional) numerical vectors. It is less obvious
		what they mean for models, which are only indirectly related by model transformation chains. Providing appropriate interpretations of these notions will make it
		possible to reuse more existing search and optimisation algorithms. Beyond that, however, there is an opportunity to explore novel search algorithms that take
		guidance from the structure and constraints encoding the search space in a model-driven context.
	
	\subsection{Model evolution}
	
		Generating new candidate solutions from existing ones is a key part of any search algorithm. In model-based search, a number of challenges need to be
		addressed:
		
		\textbf{Model breeding.}
		%
		  Many population-based algorithms rely on a notion of ``breeding'', which allows creating new candidate solutions by combining two good parent solutions. This
			is useful because it allows the search to reach new areas of the search space, hopefully benefiting from the advantages of both parent solutions. For
			example, in genetic algorithms, ``breeding'' is realised through so-called crossover operators, which combine two genes by swapping sub-sequences. While
			mutation of solutions is easily captured by model transformations as discussed, it is less clear how to express breeding. Two approaches seem worth exploring:
			
			\begin{enumerate}
			  \item \emph{Domain-specific breeders.} As with model mutation, we could use model transformations to express domain-specific breeding. These
				      transformations would take in two models and produce a new model. For example, in our \texttt{Zoo} problem, we could consider developing a
							transformation that takes two cage--animal allocations and produces a new one mixing allocations from both sources while making sure that constraints
							are not violated (e.g., updating \texttt{spaceRemaining} values and checking for \texttt{eats} relationships). Burton \emph{et al.} \cite{Burton+12}
							show a first example of this for problems where solutions are essentially sets of links between pre-existing model elements.
				\item \emph{Generic breeding through model merging.} Model breeding essentially requires identifying the common and different parts of two models so that
				      the common parts can be retained in the new solution and the different parts can be mixed suitably. This is very similar to what has been developed
							in the context of work on model differencing and model merging \cite{Kolovos09,Kolovos+09b,Maoz+10,Langer+14}. It should be possible to reuse ideas
							from this field to develop generic model breeders. The key challenge here is that mixing of differences should lead to a new model that is (a)
							different from both parent models, and (b) well-formed. This will require suitable adjustments to be made to existing diff/merge algorithms for
							models.
			\end{enumerate}
			
			It is very likely that in either case we will not be able to produce breeders that are guaranteed to produce well-formed models, introducing the need to
			deal with invalid solutions in the search. Abdeen \emph{et al.} \cite{Abdeen+14} give a good discussion of these issues in the context of genetic
			optimisation of model-transformation chains, where they use repair as well as customised ranking rules. Similar techniques could be applied to model-based
			optimisation, too.
			
		\textbf{Efficient model evolution.} 
		%
			Our current prototype randomly selects an evolver when asked to produce a new candidate solution. Should Henshin be unable to apply the rule (i.e., find no
			suitable match) we randomly select another evolver until one is applicable or we have tried all available evolvers. Especially where rules have similar 
			pre-conditions this seems an inefficient approach. We should explore mechanisms for selecting evolvers more efficiently. Denil \emph{et al.} \cite{Denil+14}
			provide some initial insights into this problem by considering optimisation algorithms to be a kind of transformation scheduling specification. This enables
			them to use different sets of evolvers at different stages of the optimisation process.
			
		\textbf{Non-deterministic matching in graph transformation engines.} 
		%
			Search-based algorithms rely on an amount of randomness underlying the exploration process. Using graph transformations as model evolvers requires the
			matching process to be non-deterministic. In other words, if there are multiple potential matches for a graph-transformation rule in a model, the choice of
			match to apply should be non-deterministic. Otherwise, we risk excluding large parts of the search space from the search as a result of an accidental
			systematic effect of how models happen to be stored in memory or of how model elements are enumerated to find potential matches. It is not clear whether
			current implementations (and in particular Henshin) are non-deterministic in this sense. If they are not, we need to explore new implementation techniques
			for balancing match efficiency and the required non-determinism.
			
		\textbf{Compact representations of solutions and evolvers.} 
		%
			Typically, a substantial part of a candidate solution will remain constant, as it essentially describes problem constraints rather than solution elements.
			Burton \emph{et al.} \cite{Burton+12} use different models to represent these static parts independently of the parts that change during search. This makes
			for a very compact solution representation, but requires a separate composition transformation whenever a solution's fitness is to be evaluated or when a
			new solution needs to be generated. There is a need to understand other similarly compact representations of candidate solutions and how they affect
			solution evolution and fitness evaluation. Similarly, we should explore ways in which evolvers can be represented more compactly and executed more
			efficiently knowing that large parts of a candidate solution never change.
		
	\subsection{Performance}
	
		Search algorithms are computationally expensive. Typically, they require a large number of iterations to be run for large populations of candidate solutions.
		Each iteration requires each candidate solution in the population to be evolved to a new solution and the fitness of these solutions to be evaluated. The
		performance of search algorithms is therefore substantially influenced by the performance of solution evolution and fitness evaluations. Depending on the size
		of the models, model transformations can be computationally expensive to execute. There has been recent interest in increasing the efficiency of model
		transformation execution \cite{Amstel+11,Meszaros2010}. We need to explore how this could be integrated into model-based optimisation. Ideally, we would like
		to be able to run model-based search even at system run time to support self-aware system adaptation.
		
	\subsection{Flexible definition of model providers}
	
		As discussed above, candidate solutions in model-based optimisation are particular in that substantial parts of the model will remain constant as they are
		describing the search problem. When using constraint solvers like Alloy \cite{Jackson02}, this would result in a large number of constraints, potentially
		impacting the performance of initial model generation. Better techniques need to be studied that can limit the performance impact on model generation.
		
	\subsection{Expressing fitness evaluations}
	
		Some fitness functions are essentially model queries, which can be efficiently expressed in languages like OCL. However, other evaluations will be more
		complex, including simulations and model analyses. At the moment, these are handled by providing a general Java interface to be implemented for specific
		fitness evaluations. Techniques better aligned with model-driven approaches need to be developed. Kessentini \emph{et al.} \cite{Kessentini+13} have made some
		interesting first proposals in this area, which need to be explored further.
		
	\section{Conclusions}
\label{section:conclusions}

  Search-based software engineering (SBSE) and model-driven engineering (MDE) are highly complementary approaches to software engineering. As a result, there has 
	been substantial interest recently in exploring the combination of both approaches, in particular using MDE technologies to simplify and streamline the 
	application of SBSE. This paper adds to the debate by
	\begin{enumerate}
		\item Providing an overview and initial classification of the current state of the art;
		\item Identifying issues with generic genetic encodings of models;
		\item Presenting an initial prototype for model-based optimisation; and
		\item Identifying a catalogue of research challenges towards complete support for model-based optimisation.
	\end{enumerate}
	
	\bibliographystyle{splncs}
	\bibliography{biblio}

\end{document}