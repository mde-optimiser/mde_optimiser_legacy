\section{Introduction}
\label{section:introduction}

	%\begin{draftlist}
		%\item Search-based and optimisation approaches to design are becoming more important (SBSE), but they are also important to adaptation (cite our cloud work,
					%mention robotics). 
		%\item The domains concerned can be quite complex, capturing them in traditional optimisation-oriented ways is difficult.
		%\item Some earlier work exists that tries to define a generic mechanism of translating models into genome representation (cite Crepe complete paper and
					%James' thesis), but this has issues. Discuss these issues in detail here or possibly in its own section below.
		%\item Main problem: loses lots of domain-specific knowledge.
		%\item We have spent lots of time to design a good meta-model based on extensive domain analysis. Why not use this knowledge when developing optimisations?
		%\item In this paper, we propose a technique and prototype tool for doing so and present its application to a simple case study. This forms the basis of a 
					%research agenda.
		%\item Structure of the paper etc.
	%\end{draftlist}

	Search-based software engineering (SBSE) is about using optimisation techniques for automating the search for (near-)optimal software designs \cite{HarmanJones01}.
	More recently, the use of search-based approaches has also been extended to software adaptation (e.g., \cite{Efstathiou+14,Chatziprimou+14}). Using search-based
	techniques allows the exploration of much larger design spaces than could be explored manually by developers. As a result, better solutions can be identified more
	quickly.
	
	However, as has been recognised before \cite{BurtonPoulding13,Kessentini+13}, the problem domains in software engineering are too complex to be effectively
	captured with traditional problem representations as they are typically used in search-based systems. Model-driven engineering (MDE) offers good techniques for
	capturing complex domains including their structural constraints by using meta-models. As a result, there has recently been increased interest in combining the
	advantages of SBSE and MDE \cite{Fleck+15,Drago+10,Drago+11,Drago+15,Efstathiou+14b,Williams13,Denil+14,Abdeen+14}.
	
	Much of this work has focused on finding good generic representations of models that are tailored towards the needs of standard optimisation algorithms (most
	typically, genetic algorithms, e.g. \cite{Deb+02}). As we will discuss in detail in Sect.~\ref{section:issues}, these generic encodings introduce their own
	challenges. Most importantly, they make it easy for search steps to produce invalid candidate solution; that is, models that do not satisfy the constraints
	expressed by the meta-model or its well-formedness rules.
	
	In this paper, we propose an alternative approach: instead of defining a secondary encoding for candidate solutions, we propose to run optimisation algorithms
	directly on models represented in standard meta-modelling data structures. We argue that given that developers have spent substantial time and effort designing
	meta-models that are a good representation of the domain, we should make use of as much of this information as possible during search and optimisation. We
	present a first prototype of a tool for running such optimisations and discuss some of the research challenges that need to be addressed to make this vision a
	reality.
	
	The remainder of this paper is strucured as follows: \draft{}