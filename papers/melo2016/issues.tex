\section{Issues with Generic Encoding of Models}
\label{section:issues}

	%\begin{draftlist}
		%\item Describe key ideas behind Crepe's approach to encoding. 
		%\item Introduce the Zoo meta-model and show an example encoding.
		%\item Then show how this is broken by simple crossover.
	%\end{draftlist}

  We are interested in optimisation of models rather than transformations. While the latter is useful when we have optimality requirements over transformation
	chains, more typically, we are interested simply in deriving optimal models. In such a case, optimising transformations incurs too much overhead in repair and
	through redundant representations of the same model through different transformation chains (effectively reducing the size of the search population).
	
	Generic genetic encodings of models as proposed in \cite{Kessentini+13,Williams13,Efstathiou+14b}, however, have their own problems. In particular, it seems very difficult to ensure locality and
	preservation of well-formedness as we will demonstrate in an example. Figure~\figref{zoo_example}~(a) shows the meta-model of a simple Zoo DSL. This DSL allows the description of zoo configurations,
	where there are cages with animals, some of which may eat other animals. An optimisation problem of interest would be to find the minimum number of cages to keep all the animals in so that there are
	no animals that eat each other. Note that cages have a maximum capacity and animals require a certain amount of space. The meta-model has been annotated to indicate how it would be encoded by the
	algorithm proposed in \cite{Williams13}. There, models are encoded as integer strings, where each model element is started by an integer identifying its meta-class (shown in dashed rectangles in 
	Fig.~\figref{zoo_example}~(a)). This is then followed by pairs of integers where the first identifies the structural feature (indicated by dashed circles in the figure) and the second provides the
	value for this feature. For associations, the values are provided by numbering all instances of the target type in sequence of appearance in the gene.
	
	\insertFigure[caption={Zoo DSL example (based on \cite{Williams13})},targetwidth=.8\textwidth]{zoo_example}
	
	Figure~\figref{zoo_example}~(b) shows an example model\footnote{We leave out the recurring \texttt{Zoo} object for simplicity.} and its encoding as a gene
	using this algorithm. To ensure smooth applicability of the standard mutation and crossover operators, genetic encodings should be local; that is, changes
	in one part should not affect other parts of the represented object. This is not the case for this generic encoding. For example, Fig.~\figref{zoo_example}~(c)
	shows the same gene after it has been split in preparation of a standard single-point crossover operation, removing the representation for object \texttt{a1}
	only. The remainder of the gene is exactly as before, but the model fragment encoded by it has changed dramatically: the eat relationship between the two animals
	has been lost and \texttt{a3} has been moved into the cage. Note that the latter change has also led to a violation of the models well-formedness rules, which
	require that \texttt{spaceRemaining} should always indicate the space left to place additional animals in a cage (and so should now be 2). We are not aware of
	any generic genetic encoding of models that has overcome this problem using generic mutation and crossover operators. As a result, such approaches require a lot
	of repair of candidate solutions, substantially worsening the optimisation performance \cite{Efstathiou+14b}.
	
	The problem could be resolved by defining domain-specific mutation and crossover operators. However, these are difficult to implement on the level of genotypes;  
	they will effectively have to constantly reinterpret the genes as the corresponding model fragments. In the next section, we propose that using optimisation
	directly on models makes it much easier to define such domain specific operators, substantially reducing the need for repair. As discussed in 
	Sect.~\ref{section:related_work}, we are not the first to propose this: Burton \emph{et al.} \cite{Burton+12,BurtonPoulding13} first proposed the idea and gave
	an example. They do not, however, provide a general implementation. Denil \emph{et al.} \cite{Denil+14} provide a first generic implementation using a
	transformation scheduling language. However, it is difficult to see how their approach would generalise to population-based algorithms as typically used for 
	multi-objective optimisation as the actual candidate-solution model is implicit in the specification. Moreover, they do not provide general mechanisms for
	encoding optimisation objectives, initial model generation, or model evolution and breeding. Therefore, there still is substantial need for further research in
	this area.