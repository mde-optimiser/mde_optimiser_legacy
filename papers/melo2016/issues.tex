\section{Issues with Generic Encoding of Models}
\label{section:issues}

	\begin{draftlist}
		\item Describe key ideas behind Crepe's approach to encoding. 
		\item Introduce the Zoo meta-model and show an example encoding.
		\item Then show how this is broken by simple crossover.
	\end{draftlist}

  We are interested in optimisation of models rather than transformations. The latter is useful when we have optimality requirements over transformation chains. More
	typically, we are interested simply in deriving optimal models. In such a case, optimising transformations incurs too much overhead in repair and through redundant
	representations of the same model through different transformation chains (effectively reducing the size of the search population).
	
	Generic genetic encodings of models as proposed in \cite{Kessentini+13,Williams13,Efstathiou+14b}, however, have their own problems. In particular, it seems very
	difficult to ensure locality and preservation of wellformedness as we will demonstrate in an example. Figure~\figref{zoo_meta_model} shows the meta-model of a
	simple Zoo DSL. This DSL allows the description of zoo configurations, where there are 
	
	\insertFigure[caption={Metamodel of a Zoo DSL (based on \cite{Williams13})},targetwidth=.8\textwidth]{zoo_meta_model}